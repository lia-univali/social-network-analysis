\section{Sistemas Embarcados}


Sistemas embarcados fazem parte de um sistema maior, normalmente atuando como seu controlador principal. Estes podem ser baseados em microprocessadores incorporados em dispositivos para monitorar e controlar as funções de seus componentes \cite{reddy:2002}. Sistemas embarcados são encontrados em uma variedade de dispositivos eletrônicos comuns de diversas áreas, como eletrônicos de consumo, eletrodomésticos, automação de escritório, equipamentos de negócios e automóveis \cite{vahid:2001}.

Estes sistemas possuem as seguintes características, definidas por \citeonline{reddy:2002}:

\begin{itemize}
    \item \textbf{Confiabilidade:} Sistemas embarcados devem ser confiáveis, pois executam funções críticas. Como por exemplo, no sistema embarcado utilizado para controle de voo, uma falha poderia causar desastrosas consequências.
    
    \item \textbf{Capacidade de resposta:} Sistemas embarcados devem responder a eventos o mais cedo possível. Por exemplo um sistema de monitoração de paciente, que deve processar os sinais do coração do paciente rapidamente e imediatamente notificar qualquer anormalidade nos sinais detectados.
    
    \item \textbf{Hardware especializado:} Já que os sistemas embarcados realizam funções específicas, são utilizados hardwares específicos. Exemplos disto são sistemas embarcados que monitoram e analisam sinais de áudio, que utilizam processadores de sinais.
    
    \item \textbf{Baixo custo:} Já que sistemas embarcados são extensamente usados em sistemas de eletrônicos de consumo, estes são sensíveis ao custo. Portanto, o seu custo deve ser baixo.
    
    \item \textbf{Robustez:} Sistemas embarcados devem ser robustos, visto que eles operam em ambientes severos. Estes devem suportar vibrações, flutuações na fonte de alimentação e calor excessivo. Devido à limitação no fornecimento de energia num sistema embarcado, a energia consumida por seus componentes deve ser minimizada.
\end{itemize}