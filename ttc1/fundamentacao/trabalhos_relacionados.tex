\section{Trabalhos Relacionados} \label{sec:relatedwork}

Nesta seção, são apresentados três trabalhos recentes com objetivos, materiais e métodos relacionados com o presente estudo. Os trabalhos incluem a pesquisa de \citeonline{Blansky2013}, que investigaram a dinâmica de contágio social entre estudantes do ensino médio, correlacionando a influência de amizades imediatas com a evolução das notas dos alunos; o estudo de \citeonline{Rambaran2017}, que analisou a influência de amizades sobre notas e frequência durante dois anos, com atenção às diferenças entre popularidade e aceitação; e a publicação de \citeonline{Gremmen2017}, cuja metodologia assemelha-se à de \citeonline{Rambaran2017}, porém com maior ênfase na diferenciação dos processos de seleção e influência de amizades sobre o desempenho acadêmico.

Ao término da seção, apresenta-se uma análise comparativa entre os três trabalhos selecionados e o presente, fornecendo uma visão geral acerca das diferenças e similaridades nas abordagens.

\subsection{Spread of Academic Success in a High School Network} \label{sec:blansky}

O estudo de \citeonline{Blansky2013} envolveu a análise do processo de disseminação de comportamento em alunos de uma escola dos Estados Unidos, investigando o impacto deste processo nas notas dos estudantes ao longo do tempo. A pesquisa foi realizada na escola Maine-Endwell High School (Nova Iorque) no ano de 2011, sobre estudantes da turma do décimo primeiro estágio\footnote{A turma é denominada \textit{eleventh grade} nos Estados Unidos, sendo equivalente ao segundo ano do Ensino Médio no Brasil.}.

A amostra utilizada no estudo foi composta por 158 alunos, onde cada integrante teve de avaliar cada um de seus colegas em termos da proximidade social entre os dois, em uma escala de: (i) melhor amigo; (ii) amigo; (iii) conhecido; (iv) não nos conhecemos\footnote{No artigo original, os termos utilizados foram, em ordem, \textit{best friend}, \textit{friend} e \textit{acquaintance}.}. O questionário também perguntava se os alunos pertenciam à mesma família, porém esta informação não foi utilizada para compôr a rede complexa que modela os alunos amostrados. Juntamente com este dado, os autores obtiveram, diretamente da escola, as notas médias de cada aluno em dois pontos no tempo: na data de aplicação do questionário e exatamente um ano depois da data.

Com ambas as fontes de dados associadas, os autores construíram um modelo estatístico utilizando o método de regressão linear, identificando uma correlação positiva e estatisticamente significante entre o aumento do desempenho acadêmico de alunos e a diferença de suas notas com a de seus amigos; assim, alunos cujos amigos possuem notas superiores às suas tendem a tirar notas cada vez maiores.

Para eliminar a influência de fatores indevidos, os autores construíam um modelo nulo da rede complexa, seguindo a mesma linha de raciocínio da Modularidade (Seção \ref{sec:modularity}) de \citeonline{Newman2004}. Após analisar 500 instâncias do modelo nulo, os autores observaram que o grau de correlação diminuiu significativamente em todas elas, confirmando que notas dos amigos imediatos dos estudantes de fato causam impacto em seu desempenho.

\subsection{Academic Functioning and Peer Influences: A Short-Term Longitudinal Study of Network--Behavior Dynamics in Middle Adolescence} \label{sec:rambaran}

Em sua pesquisa, \citeonline{Rambaran2017} analisaram a relação entre laços de amizade e o funcionamento acadêmico de adolescentes, o qual é composto tanto pelas notas médias quanto pelo absenteísmo dos estudantes. Como objetivos principais, os autores elencaram: (i) verificar relação entre funcionamento acadêmico e processos de seleção de influência de pares; (ii) verificar a direção da operação dos processos de seleção e influência; e (iii) analisar o impacto da popularidade e da aceitação entre amigos sobre a associação entre influência de pares e funcionamento acadêmico.

Foi realizada uma amostragem sobre uma escola pública de ensino médio, localizada no sul da Califórnia, onde 342 dos 1200 alunos matriculados participaram do estudo. A coleta de dados foi realizada de forma longitudinal, iniciando em setembro de 1997 e sendo concluída em junho de 1999. Quatro repetições foram realizadas, com intervalos de cerca de 6 meses, sendo que os mesmos indicadores foram utilizados em todas as instâncias.

Assim como no trabalho de \citeonline{Blansky2013}, os autores utilizaram um formulário onde os alunos avaliam seus colegas; porém, devido ao elevado número de integrantes, cada aluno avaliou apenas 50 colegas, escolhidos aleatoriamente. O formulário foi composto por duas perguntas por indivíduo: ``quão popular você considera este aluno?'' e ``quanto você gosta de bater papo\footnote{A expressão inglesa \textit{hang out} foi utilizada no formulário original.} com este aluno?''. Ambas as perguntas possuiam uma escala no intervalo de 1 a 5. Além disso, cada aluno recebeu uma folha com os nomes de todos os participantes, sendo solicitado que circulassem os nomes de todos os amigos próximos. Foi associado a cada respondente seu histórico de notas médias e o número de faltas não justificadas.

No estágio de análise de resultados, os autores utilizaram o sistema SIENA, descrito em detalhes na Seção \ref{sec:siena}, que fornece um conjunto de ferramentas para análise longitudinal de redes sociais, envolvendo um modelo de simulação baseado em atores. Para contemplar o processo de seleção de pares, os autores utilizaram as nomeações de amigos próximos por escrito; para o processo de influência, foi analisada a tendência das notas de pares se tornarem mais próximas ao longo do tempo.

Como conclusões, os autores reportaram que indivíduos que formam laços de amizade com alunos que demonstram alto desempenho tendem a tirar notas mais altas, enquanto aqueles que tornam-se amigos de alunos de baixo desempenho tendem a faltar mais e, por consequência, tirar notas mais baixas. Além disso, indivíduos com baixa frequência tendem a receber menos menções de amizade próxima, principalmente de alunos com alto desempenho. Indivíduos com alta popularidade causaram impacto apenas em amizades mútuas e apenas sobre frequência, enquanto alunos com alta taxa de aceitação (dada pelo indicador ``gosta de bater papo'') causaram impacto tanto em amizades unilaterais quanto mútuas e tanto em notas quanto frequência.
% Esclarecer que unilateral = díade e mútua = tríade

\subsection{First Selection, Then Influence: Developmental Differences In Friendship Dynamics Regarding Academic Achievement} \label{sec:gremmen}

Em um estudo consideravelmente similar ao de \citeonline{Rambaran2017}, \citeonline{Gremmen2017} exploram a relação entre os processos de seleção e influência e o desempenho de estudantes em duas escolas da Holanda. Os autores selecionaram alunos na fase de transição entre o ensino fundamental e o ensino médio holandês, o qual é baseado em três trilhas de estudo; com isso, foi possível capturar com maior acurácia o processo de seleção de pares (pois o aluno é inserido em um ambiente com um conjunto de colegas diferente), diferenciando-o da influência, que se dá após laços de amizade terem sido formados.

Como objetivos do trabalho, os autores buscaram investigar duas hipóteses: (i) a probabilidade de dois indivíduos se tornarem amigos aumenta conforme maior for a similaridade de seu desempenho acadêmico; e (ii) o desempenho acadêmico de amigos tende a se tornar mais similar ao longo do tempo. A primeira trata do processo de seleção, enquanto a segunda trata da influência.

Quanto aos dados, o estudo considerou os dois primeiros anos do ensino médio holandês, envolvendo um total de 556 participantes. A coleta de dados foi realizada em 6 datas, estendendo-se de outubro de 2011 a abril de 2013. Os indicadores utilizados foram: notas médias, melhores amizades na classe (por nomeação; ou seja, 0 ou 1), assiduidade com deveres de casa (1 a 10) e satisfação com a escola (1 a 4). O primeiro foi obtido diretamente com as escolas, enquanto os demais foram coletados através de um questionário.

Assim como \citeonline{Rambaran2017}, os autores utilizaram o sistema SIENA para analisar os dados coletados. As notas foram divididas em três grupos temáticos: (i) idioma, composto pelas disciplinas de holandês e inglês; (ii) exatas/ciências, composto por matemática e biologia; e (iii) social, composto por história e geografia.

Em sua análise, os autores identificaram que, no primeiro ano do ensino médio holandês, a seleção de amizades é afetada significativamente pela similaridade de notas no mesmo grupo temático, mas não foi possível identificar o processo de influência. No segundo ano, entretanto, o comportamento inverso foi observado: a influência de amizades apresentou grande efeito no desempenho acadêmico, mas a seleção se deu de forma independente da similaridade de notas.

Além disso, alunos com alto desempenho tendem a evitar formar laços de amizade com colegas com baixo desempenho; no sentido oposto, o comportamento também pode ser observado, mas com menos proeminência. Por fim, tanto a seleção quanto a influência se mostraram mais evidentes quando notas do mesmo grupo temático foram consideradas, ao invés da média geral.

\subsection{Análise Comparativa} \label{sec:relatedcomparison}

Esta seção apresenta uma análise comparativa entre os trabalhos de \citeonline{Blansky2013}, \citeonline{Rambaran2017} e \citeonline{Gremmen2017}, expondo de forma sintética suas similaridades, diferenças e evidenciando a maneira com que o presente estudo se relaciona com eles.

\begin{quadro}[ht]
    \caption{Comparativo entre trabalhos relacionados}
    \label{board:related}
    \fontsize{10}{12}\selectfont
    \def\arraystretch{1.25}
    \begin{tabularx}{\textwidth}{|c|Y|c|c|c|}
        \hline
        \textbf{Trabalho} & \textbf{Participantes} & \textbf{Escala de amizade} & \textbf{Período de estudo} & \textbf{Validação}
        \\ \hline
        \citeonline{Blansky2013} & 2º ano do ensino médio, EUA & 0 a 3 & 1 ano & Modelo nulo
        \\ \hline
        \citeonline{Rambaran2017} & 1º e 2º anos do ensino médio, EUA & 1 a 5; 0 ou 1 & 2 anos & SIENA
        \\ \hline
        \citeonline{Gremmen2017} & 1º e 2º anos do ensino médio, Holanda & 0 ou 1 & 2 anos & SIENA
        \\ \hline
        Este estudo & Ensino superior, Brasil & 0 a 5 & 1 ano & Modelo nulo
        \\ \hline
    \end{tabularx}
\end{quadro}

Todos os trabalhos descritos realizam a reconstrução das redes sociais através da aplicação de questionários, respondidos pelos próprios alunos, cada um com uma escala própria. \citeonline{Rambaran2017} utilizam duas questões relacionadas a amizade, conforme descrito na Seção \ref{sec:rambaran}; portanto, na tabela, são apresentadas as escalas referentes às questões ``bater papo'' e ``amigos mais próximos'', respectivamente. Quanto ao período de estudo, no caso do trabalho de \citeonline{Blansky2013} e do presente, apenas uma coleta de dados sobre amizade via questionário será realizada, porém um histórico extenso de notas será levado em consideração.